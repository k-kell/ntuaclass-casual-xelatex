\documentclass{ntuaclass} % η κλάση του εγγαφου
\usepackage{lipsum} % το χρησιμοποιώ για dummy-text

% ο φάκελος με τις εικόνες
\graphicspath{{./images/},{./plots/}}

% το αρχείο με τις βιβλιογραφικές αναφορές  uncomment για βιβλιογραφία
\bibliography{bibliography.bib} 

% ο τίτλος του εγγράφου
\Type{τέστ και τεστ}
\Title{Τίτλος}
\Eks{7}
\Mathima{Ρευστά}
\Onomatepwnumo{Kellaris}
\Armhtrwou{02117010}
\Omada{7} % προαιρετικό 
\Hmeromhnia{\today}

% διορθώνει τα μεταδεδομένα του αρχείου
\fixmetadata

%αλλαγή τίτλου του κεφαλαίου π.χ. ερώτημα
%\setchapname{Ερώτημα} uncomment για χρήση

% χρήση της εντολής ώστε να κάνει compile μόνο το κεφάλαιο που αλλάξαμε
%\includeonly{chapters/3rdchapter} uncomment για χρήση

\begin{document}

% δημιουργεί το εξώφυλλο
\ntuatitle

%δημιουργεί τον πίνακα περιεχομένων με σωστό τρόπο
\maketoc

% μια καλή πρακτική είναι κάθε κεφάλαιο να γράφεται σε ξεχωριστό αρχείο
\section{Πρώτο δοκιμαστικό κεφάλαιο}
\lipsum
\cite{bowenspaper}
\begin{figure}[tbp]
    \centering
    \input{plots/gplot_example.tex}
    \caption{Ένα διάγραμμα με \en{gnuplot}}
\end{figure}

\section{Δοκιμαστικό κεφάλαιο με εικόνα svg}
\subsection{πρώτη ενότητα}
\lipsum[1-2]
\svg{images/ress.pdf_tex}{Δοκιμαστικός υπότιτλος}{1}
\subsection{άλλη μια ενότητα}
\lipsum[3-4]
\section{Kεφάλαιο με αγγλικά στον τίτλο και pdf}
\pdf{a3.pdf}{testcaption}}{2}
\subsection{τελευταία ενότητα}
\lipsum
\begin{figure}[tbp]
\centering
\begin{tikzpicture}
    \begin{axis}[
    siunitxlabels,
    legend style={minimum width=8 cm},
    xmin=0,xmax=4000,
    ymin=0,ymax=750,
    xtick={0,500,1000,1500,2000,2500,3000,3500,4000},
    ytick={0,150,300,450,600,750},
    xlabel={Απόσταση $m$},
    ylabel={Συγκέντρωση $μg/m^3$},
    legend pos=north east,
    ymajorgrids=true,
    xmajorgrids=true,
    grid style=dashed,
    width=0.9\textwidth,
    height=10cm
    ]
    \addplot[color=blue,no marks]  
      table[x=Distance, y=Concentration ,col sep=comma]
      {plots/5_0.2_0.77_0.24.csv };

    \addplot[color=red,no marks]  
      table[x=Distance, y=Concentration ,col sep=comma]
      {plots/10_0.2_0.77_0.24.csv };

    \addplot[color=green,no marks]  
      table[x=Distance, y=Concentration ,col sep=comma]
      {plots/15_0.2_0.77_0.24.csv };
    \addlegendentry{H = 5m - $z_0 = 0.2m$ - bowen = 0.77 - albedo = 0.16}
    \addlegendentry{H = 10m - $z_0 = 0.2m$ - bowen = 0.77 - albedo = 0.16}
    \addlegendentry{H = 15m - $z_0 = 0.2m$ - bowen = 0.77 - albedo = 0.16}
    \end{axis}
  \end{tikzpicture}
    \caption{Γράφημα συγκεντρώσεων με παράμετρο το ύψος καμινάδας}
  \label{fig:default}
\end{figure}
\lipsum

% uncomment για βιβλιογραφία

%\addcontentsline{toc}{chapter}{Βιβλιογραφία}
\printbibliography[
heading=bibintoc, title={Βιβλιογραφία}]
\end{document}